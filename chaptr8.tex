\chapter[آشنایی با Apt]{آشنایی با \lr{Apt}}
\lr{Apt} (مخفف \lr{Advanced Packaging Tool})، برنامهٔ نصب و حذف نرم‌افزارها در توزیع دبیان گنو/لینوکس است. از آن‌جایی که اوبونتو از دبیان مشتق شده است، اوبونتو نیز \lr{Apt} را به همراه دارد. نرم‌افزارهایی مثل \lr{Ubuntu Software Center} و \lr{Synaptic} هم تنها رابطی گرافیکی برای \lr{Apt} اند. پس آشنایی با \lr{Apt}، می‌تواند در کنترل بیش‌تر بر سیستم به ما کمک کند.

\section{لیست مخازن}
برای این که \lr{Apt} کار کند، به لیست مخازن نیاز دارد. لیست مخازن، شامل آدرس مکان‌هایی است که می‌توان از آن برای اوبونتو نرم‌افزار تهیه کرد. یکی از تفاوت‌های اصلی ویندوز و گنو/لینوکس نیز همین است. در اوبونتو به احتمال خیلی زیاد، به هیچ‌گونه \lr{CD} و \lr{DVD}ای برای نصب نرم‌افزارها نیاز نخواهید داشت. حتا اغلب اوقات هم لازم نیست برای نصب یک نرم‌افزار، به دنبال فایل نصب‌اش در اینترنت بگردید. بیش‌تر نرم‌افزارهای مورد نیاز در مخازن رسمی اوبونتو موجودند.\\
مخازن رسمی اوبونتو، روی اینترنت‌اند. هرچند تمام نرم‌افزارهای مخازن اوبونتو بر روی چند \lr{DVD} هم قابل تهیه است، اما باید توجه داشت که نرم‌افزارهای مخازن همیشه در حال به‌روزآوری‌اند. پس برای استفاده از جدیدترین نسخه‌های نرم‌افزارها، به اینترنت نیازمندید. البته حجم نرم‌افزارهای اوبونتو (و کلاً گنو/لینوکس‌ها)، به مراتب از ویندوز کم‌تر است. دلیل این موضوع، استفاده‌کردن نرم‌افزارهای گنو/لینوکسی از ابزارها و کتاب‌خانه‌های مشترک است.\\
\subsection{محل لیست مخازن}
لیست مخازن در گنو/لینوکس‌های بر پایهٔ دبیان (شامل اوبونتو)، یک فایل متنی به نام \lr{sources.list} که در مسیر \lr{\texttt{/etc/apt/sources.list}} است. برای ویرایش این فایل، باید فایل را در ویرایشگرهای متن باز کنید. در اوبونتو، دو ویرایشگر \lr{gedit} و \lr{Nano} وجود دارد که به ترتیب گرافیکی و متنی‌اند. کار با \lr{gedit} بسیار راحت‌تر از \lr{Nano} است؛ پس فایل را با \lr{gedit} باز می‌کنیم. اما ویرایش‌کردن فایل لیست مخازن، نیازمند اجازهٔ ریشه است؛ در نتیجه، راحت‌ترین راه بازکردن این فایل با مجوز ریشه، استفاده از دستور \lr{\texttt{sudo gedit /etc/apt/sources.list}} است.\\
بعد از زدن دستور، پنجره‌ای مانند پنجرهٔ زیر باز می‌شود.\\
تعدادی خط را می‌بینید. اگر خطی در ابتدایش، علامت \lr{\texttt{\#}} وجود داشته باشد، غیرفعال است. بقیه فعال‌اند و \lr{Apt} آن‌ها را می‌خواند.

\subsubsection{مفهوم خطوط لیست مخازن}
هر خط در این فایل، شامل ۴ بخش به شکل \lr{\texttt{deb address distro component1 component2}} است. بخش اول، یعنی «\lr{\texttt{deb}}»، مشخص می‌کند که آرشیو مورد نظر، دارای فایل‌های نصب با پسوند \lr{deb} است. در این بخش، به جای «\lr{\texttt{deb}}»، «\lr{\texttt{deb-src}}» هم می‌تواند قرار بگیرد که یعنی آرشیو، دارای فایل‌های منبع است، نه فایل‌های نصب دبیانی.\\
در بخش دوم یا «\lr{\texttt{address}}، باید آدرس مخزن را وارد کنید که می‌تواند آدرسی اینترنتی یا آدرسی محلی و روی کامپیوتر یا شبکهٔ خانگی‌تان باشد؛ اما اکثراً این یک آدرس اینترنتی است.\\
در بخش «\lr{\texttt{distro}}»، نام توزیع کنونی‌تان را وارد کنید. مثلاً برای اوبونتوی ۱۳/۱۰، باید «\lr{\texttt{raring}} نوشته شود.\\
در آخرین قسمت هم، نوع مخزن را وارد می‌کنید. اوبونتو مخازن مختلفی به نام‌های «\lr{main}»، «\lr{universe}»، «\lr{multiverse}»، «\lr{restricted}» و \ldots دارد.\\
در بخش آخر، می‌توان چندین نوع مخزن را وارد کرد. یعنی بعد از قسمت سوم، هر چه که وارد شود، مربوط به نوع مخزن خواهد بود.

\section[دستورهای معمول و اصلی Apt]{دستورهای معمول و اصلی \lr{Apt}}
\lr{Apt} نام یک ابزار است و اصولاً دستوری به شکل \lr{\texttt{apt}} وجود ندارد. برای استفاده از ابزار \lr{Apt}، باید از دستورهای زیرمجموعهٔ آن، مثل \lr{\texttt{apt-get}} و \lr{\texttt{apt-cache}} استفاده کرد. دستور \lr{\texttt{apt-get}}، بیش‌ترین استفاده را برای ما دارد.

\subsection[apt-get]{\lr{apt-get}}
همان‌گونه که گفته شد، دستور \lr{\texttt{apt-get}} مهم‌ترین دستور است. چون دستور برای تغییر در بعضی فایل‌ها و پوشه‌های کل سیستم تغییر ایجاد می‌کند، برای استفاده از آن، باید کاربر ریشه بود (یعنی باید با \lr{\texttt{sudo}} همراه شود.\\
از این دستور برای کارهای زیر استفاده می‌شود.

\begin{description}
\item [به‌روزآوری لیست نرم‌افزارهای مخازن]: با به کار بردن دستور \lr{\texttt{sudo apt-get update}}
\item [نصب نرم‌افزار]: با دستور \lr{\texttt{sudo apt-get install software}} که به جای \lr{\texttt{software}}، باید نام نرم‌افزار مورد نظر خود را بنویسید. اگر حجم فایل‌هایی که قرار است دانلود شود، زیاد باشد، پیامی مبنی بر تایید دانلود و نصب نرم‌افزار ظاهر می‌شود که با زدن دکمهٔ \lr{Enter}، تایید می‌شود.\\
از این دستور برای نصب نسخهٔ جدید نرم‌افزار هم می‌توان استفاده کرد.
\item [حذف نرم‌افزار]: با دستور \lr{\texttt{sudo apt-get remove software}} نرم‌افزار حذف می‌شود، اما فایل‌های پیکربندی آن روی سیستم باقی می‌ماند. برای حذف نرم‌افزار همراه با حذف فایل‌های پیکربندی آن، از دستور \lr{\texttt{sudo apt-get purge software}} استفاده کنید.
\item [آپدیت‌کردن همهٔ نرم‌افزارها]: برای این کار، از دستور \lr{\texttt{sudo apt-get upgrade}} استفاده کنید.
\item [آپگرید به نسخهٔ جدید اوبونتو]: این کار با آپدیت‌کردن نرم‌افزارها متفاوت است. با آپگرید، نسخهٔ اوبونتو عوض می‌شود و بعد از آپگرید، از مخازن نسخهٔ جدید اوبونتو که زودتر آپدیت می‌شوند، استفاده می‌شود. برای آپگرید، از دستور \lr{\texttt{sudo apt-get dist-upgrade}} استفاده کنید.
\item [دانلود بسته‌ها]: برای دانلود بسته‌ها بدون نصب آن‌ها در پوشهٔ کنونی، از \lr{\texttt{sudo apt-get download software}} استفاده کنید.
\end{description}
