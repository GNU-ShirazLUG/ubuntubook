\chapter{نرم‌افزارهای برتر اوبونتو}
در این بخش، به معرفی برترین و کاربردی‌ترین نرم‌افزارهای اوبونتو می‌پردازیم و توضیح مختصری راجع به هر یک از نرم‌افزارها ارایه می‌کنیم. لازم به ذکر است که تمامی نرم‌افزارهای زیر، آزاد، متن‌باز و رایگان بوده و شما می‌توایند این نرم‌افزارها را به راحتی و با جست‌و‌جو در \lr{Software Center} نصب کنید.

\section[libreOffice]{\lr{LibreOffice}}
لیبره آفیس یکی از اولین نیازمندی‌های کاربر متوسط است. این بستهٔ نرم‌افزاری، جایگزین مناسبی برای نرم‌افزار آفیس مایکروسافت است و از بخش‌های زیر تشکیل می‌شود:
\lr{Power Designer}
\begin{itemize}
\item \lr{\textbf{Writer}}: برنامه‌ای است برای نوشتن و ویرایش متن. این نرم‌افزار، زبان فارسی را کاملاً پشتیبانی می‌کند. خروجی پیش‌فرض آن، \lr{odt} است اما می‌توانید خروجی هایی مانند \lr{doc} و \lr{pdf} نیز داشته باشید.
\item \lr{\textbf{Impress}}: نرم‌افزار ساخت فایل‌های ارائه که معادل \lr{PowerPoint} در مجموعهٔ آفیس مایکروسافت است.
\item \lr{\textbf{Calc}}: این نرم‌افزار برای ساخت و ویرایش فایل‌های صفحه‌گسترده است.
\item \lr{\textbf{Draw}}: برای طراحی های ساده گرافیکی مورد استفاده قرار می گیرد.
\item \lr{\textbf{Base}}: نرم افزاری برای طراحی مفهومی پایگاه‌داده و روابط بین جداول است و عملکردی مانند \lr{MS Access} و \lr{Power Designer} دارد.
\item \lr{\textbf{Math}}: کار نوشتن فرمول‌های ریاضی را انجام می‌دهد.
\end{itemize}


