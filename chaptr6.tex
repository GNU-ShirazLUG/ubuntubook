\chapter{نرم‌افزارهای معادل}
از تمام مزایای لینوکس مثل آزادی که بگذریم، شما در گنو/لینوکس هم باید کارهای متداول خود را انجام بدهید. در لیست زیر، نرم‌افزارهای گنو/لینوکسی معادل نرم‌افزارهای پرکاربرد در ویندوز و \lr{Mac OS X} معرفی می‌شوند.\\

\begin{table}[ht]
\caption{لیست نرم‌افزارهای معادل}
\centering
\begin{tabular}{|c|c|}
\hline
\textbf{\lr{\Large Ubuntu}} & \textbf{\lr{\Large Windows / Mac OS X}} \\[1ex]
\hline
\lr{Pinta} & \lr{Paint}\\
\hline
\lr{VLC} & \lr{KMPlayer}\\
\hline
\lr{Totem} & \lr{Windows Media Player}\\
\hline
\lr{Gimp} & \lr{Photoshop}\\
\hline
\lr{OpenShot, PiTiVi} & \lr{Windows Media Player}\\
\hline
\lr{Rhythmbox, Noise} & \lr{iTunes}\\
\hline
\lr{gedit} & \lr{Windows Notepad}\\
\hline
\lr{Blender} & \lr{Autodesk 3D Max}\\
\hline
\lr{LibreCAD} & \lr{Autodesk AutoCAD}\\
\hline
\lr{Audacious} & \lr{Winamp}\\
\hline
\lr{Evince} & \lr{Adobe Acrobat Reader}\\
\hline
\lr{Inkscape} & \lr{Adobe Illustrator}\\
\hline
\lr{Scribus} & \lr{Adobe InDesign}\\
\hline
\lr{LibreOffice} & \lr{Microsoft Office, Apple iWork}\\
\hline
\lr{Empathy, Pidgin} & \lr{Yahoo Messenger, Google Talk}\\
\hline
\end{tabular}
\end{table}
